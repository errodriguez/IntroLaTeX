\documentclass[10pt,letterpaper]{article}
\usepackage[utf8]{inputenc}
\title{Listas}
\author{Curso de introducción a LaTeX}
\begin{document}
\maketitle

Existen tres tipos de listas que podemos emplear en \LaTeX: listas con viñetas, listas enumeradas y listas de descripción.\\

\noindent Ejemplo de una lista con viñetas:
\begin{itemize}
   	\item Elemento 1		
    \item Elemento 2
   	\item Elemento 3
\end{itemize}

\noindent Ejemplo de una lista enumerada:
\begin{enumerate}
    \item Aritmética
    \item Álgebra
    \item Trigonometría
\end{enumerate}

\noindent Una lista con sublistas: 
\begin{itemize}
   \item Elemento 1
   \item Elemento 2
           \begin{itemize}
               \item Subelemento 1
           \end{itemize}
    \item Elemento 3
\end{itemize}

\noindent Una lista enumerada con sublistas: 
\begin{enumerate}
    \item Aritmética
    \item Álgebra
       	\begin{enumerate}
   	    	\item Números enteros
   	    	\item Jerarquía de operaciones
       	\end{enumerate}
    \item Trigonometría
\end{enumerate}

\noindent Las listas de descripción, por otro lado, son un tipo de lista especialmente diseñados para escribir diccionarios y glosarios de términos, por ejemplo:

\begin{description}
  \item[Axioma] Proposición que se considera ``evidente'' y se acepta sin requerir demostración previa.
  \item[Teorema] Proposición que afirma una verdad demostrable.
  \item[Corolario] (del latín \emph{corollarium}) es un término que se utiliza en matemáticas y en lógica para designar la evidencia de un teorema o de una definición ya demostrados, sin necesidad de invertir esfuerzo adicional en su demostración.
\end{description}

\noindent Ahora crea una lista con viñetas de tres elementos:

\end{document}