\documentclass[10pt,letterpaper]{article}
\usepackage[utf8]{inputenc}
\title{Tablas}
\author{Curso de introducción a LaTeX}
\begin{document}
\maketitle
Primero desplegamos una tabla simple: \\

\begin{tabular}{rcl}
    1 & 2 & 3 \\
    4 & 5 & 6 \\
    7 & 8 & 9 \\
	10 & 11& 12\\
\end{tabular}\\ %Donde, {lll} especifica que queremos una tabla con 3 columnas y que su texto esté justificado a la izquierda

Después, le añadimos separadores verticales: \\

\begin{tabular}{ l | c || r }
  1 & 2 & 3 \\
  4 & 5 & 6 \\
  7 & 8 & 9 \\
  \end{tabular}\\ %Donde, {l | c || r} especifica que queremos una tabla con 3 columnas, la primera justificada a la izquierda, la segunda al centro y la tercera a la derecha, y que habrá dos separadores verticales entre las columnas, uno con una línea simple, y otro con una doble línea

\hrule

Ahora, crearemos una tabla con líneas horizontales: \\

\begin{tabular}{rcl}
\hline
Autores & Año   & Título \\
\hline
Galván Magaña Felipe & 1989  & El alimento de los peces de pico \\
Ochoa Báez Rosa Isabel & 1989  & Los picudos \\
Hernández Carmona Gustavo & 1990  & Investigación en algas marinas... \\
\hline
\end{tabular} \\ %Donde, el comando \hline antes de cada línea de la tabla introduce una línea horizontal que se utilizaa como separador

Una tabla con bordes se construiría así: \\

\begin{tabular}{|l|l|l|}
    \hline
    Año de nacimiento & Nombre & Nivel SNI \\
    \hline
    1982 & Sandra Rodríguez & 2 \\
    \hline
    1984 & Raúl Estrada & 1 \\
    \hline
    1989 & Erika Martínez & Candidato \\
    \hline    
\end{tabular} \\%Nótese que en este caso, la última fila necesita la doble barra invertida \\ ya que hay un elemento debajo, \hline. 

Ahora, añade a este ejemplo una nueva tabla con 3 columnas y 3 renglones, que utilice una línea horizontal al inicio y al pie de la tabla:

\end{document}