\documentclass[letterpaper,11pt]{article}
\usepackage[utf8]{inputenc}
\usepackage{babel}
\usepackage[usenames,dvipsnames]{xcolor} %Para soportar los colores extendidos
\parskip=2mm
\begin{document}

\begin{center}
	{\Huge {\color{blue}Star Trek}}               \\
    Curiosidades sobre la propulsión Warp \\
   Eduardo René Rodríguez Ávila
\end{center}

El \textit{empuje warp} (empuje por curvatura; también conocido como ``impulso de deformación'' o ``de distorsión'') es una forma teórica de propulsión su\-per\-lu\-mí\-ni\-ca. Este empuje permite propulsar una nave espacial a una velocidad equivalente a varios múltiplos de la velocidad de la luz, mientras se evitan los problemas asociados con la dilatación relativista del tiempo. Este tipo de propulsión se basa en curvar o distorsionar el Espacio-tiempo, de 
tal manera que permita a la nave ``acercarse'' al punto de destino. El empuje por curvatura 
no permite, ni es capaz de generar, un viaje instantáneo entre dos puntos a una velocidad 
infinita, tal y como ha sido sugerido en algunas obras de ciencia ficción, en las que se 
emplean tecnologías imaginarias como el ``hipermotor'' o ``motores de salto''.

	\section{La invención del motor Warp}

En la historia de \textbf{Star Trek} se reconoce que el motor de curvatura fue inventado, en la Tierra, por Zefram Cochrane. La película Star Trek: Primer Contacto muestra como, en el año 2063, Cochrane realiza el primer viaje de curvatura de la especie humana, usando un antiguo misil nuclear intercontinental, modificado para viajar en el espacio y, una vez ahí, generar una 
burbuja Warp. Cochrane, para crear la burbuja Warp alrededor de la nave -y distorsionar el 
Espacio-tiempo para su desplazamiento- precisó de una inmensa cantidad de energía (que 
obtuvo gracias a la reacción entre matería-antimateria). Este primer viaje supuso un hito, 
permitió alcanzar un factor de curvatura de 1,0 y condujo directamente al primer contacto 
con una raza extraterreste: los vulcanos.


	\section{Velocidad de curvatura. Factor de curvatura}

La unidad empleada con la velocidad de curvatura es el \textit{factor de curvatura} (``warp factor''). La equivalencia entre factores de curvatura obtenidos por los reactores warp y velocidades medidas en múltiplos de la velocidad de la luz es en cierto modo ambigua.

Según la guía para escritores de episodios de Star Trek de la Serie Orig\-inal, los factores 
warp se obtienen mediante la aplicación de la siguiente fórmula cúbica:

\begin{center}
    $s(w) = w^3c$
\end{center}

donde $w$ es el factor warp, $s(w)$ es la velocidad medida en el espacio normal y $c$ es la 
velocidad de la luz. Según esta fórmula, ``warp 1'' es equivalente a la velocidad de la 
luz, "warp 2" equivale a 8 veces la velocidad de la luz, "warp 3" equivale a 27 veces 
la velocidad de la luz, etc.

\subsection{Factor Warp, velocidad y tiempo de viaje a Alpha Cen\-tau\-ry}

\begin{tabular}{ccc}
	\hline
	Factor curvatura & Velocidad (múltiplos de c) & Días de viaje \\
	\hline
	5  & 125 &  12.5   \\
	8  &  512 &  4.1    \\
	9.5 &   857 & 2.5 \\
	\hline
\end{tabular}

\end{document}