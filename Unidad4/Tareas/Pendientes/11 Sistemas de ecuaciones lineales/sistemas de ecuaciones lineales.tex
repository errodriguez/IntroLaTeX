\documentclass[12pt,letterpaper]{article}
\usepackage[utf8]{inputenc}
\usepackage[spanish, mexico]{babel} %Para que el contenido del documento esté en español de México
\usepackage[hidelinks]{hyperref} %Para el enlace al final del documento
\usepackage{amsmath} %Para soportar el entorno matrix
\usepackage{amsfonts} %Para soportar el comando de formato matemático \mathbb
\title{}
\author{Tu nombre}
\begin{document}
Sistema de ecuaciones lineales

En matemáticas y álgebra lineal, un sistema de ecuaciones lineales, también conocido como sistema lineal de ecuaciones o simplemente sistema lineal, es un conjunto de ecuaciones lineales (es decir, un sistema de ecuaciones en donde cada ecuación es de primer grado), definidas sobre un cuerpo o un anillo conmutativo. Un ejemplo de sistema lineal de ecuaciones sería el siguiente:


    2x1 + x2  =  1
    x1 + x2 = 4

El problema consiste en encontrar los valores desconocidos de las variables x1 y x2 que satisfacen las dos ecuaciones.

El problema de los sistemas lineales de ecuaciones es uno de los más antiguos de la matemática y tiene una infinidad de aplicaciones, como en procesamiento digital de señales, análisis estructural, estimación, predicción y más generalmente en programación lineal así como en la aproximación de problemas no lineales de análisis numérico.

En general, un sistema con m ecuaciones lineales y n incógnitas puede ser escrito en forma normal como:


		a11x1 + a12x2  + ...  + a1nxn  = b1
		a21x1 + a22x2  + ...  + a2nxn  = b2
		...      ...     ...     ...    ...
		am1x1 + am2x2  + ...  + amnxn = bm

Donde x1,...,xn son las incógnitas y los números  $ a_{ij} \in \mathbb{K} $  son los coeficientes del sistema sobre el cuerpo $\mathbb{K} [= R, \mathbb{C}, ...]$. Es posible reescribir el sistema separando con coeficientes con notación matricial:

     a11   a12   ...   a1n 
     a21   a22   ...   a2n
     :      :     :.   :
     am1   am2    ... amn
      
     
     x1 x2 ... xn 
     =
     x1 x2 ... xn 
      = 
     b1 b2 ... bm


Si representamos cada matriz con una única letra obtenemos:


Ax = b

Donde A es una matriz m por n, x es un vector columna de longitud n y b es otro vector columna de longitud m. El sistema de eliminación de Gauss-Jordan se aplica a este tipo de sistemas, sea cual sea el cuerpo del que provengan los coeficientes. La matriz A se llama matriz de coeficientes de este sistema lineal. A b se le llama vector de términos independientes del sistema y a x se le llama vector de incógnitas.

Más información en: \url{http://es.wikipedia.org/wiki/Sistema_de_ecuaciones_lineales}
\end{document}