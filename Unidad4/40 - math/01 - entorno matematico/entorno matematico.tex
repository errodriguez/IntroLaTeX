\documentclass[10pt,letterpaper]{article}
\usepackage[utf8]{inputenc}
\title{El entorno matemático}
\author{Curso de \LaTeX}
\begin{document}
\maketitle

Para escribir expresiones matemáticas en \LaTeX\ tenemos que capturarlas dentro del entorno matemático.

Este entorno matemático puede aparecer dentro de las líneas de nuestro texto, o en un entorno de visualización especial, que estará separado del resto del texto. 

Para incluir una expresión matemática en una línea del texto, colocamos nuestra expresión matemática dentro de los símbolos \$ ... \$, que son los delimitadores del entorno matemático en línea con el texto, tal y como se muestra en el siguiente ejemplo:

... si $x=0$ entonces $y^2 = 4p+7$, pero si damos otro valor a $x$ no se que pase ... 

Noten que la tipografía aplicada las literales $x$ y $y$ es distinta a la de un texto común (x y).

%Nota: los delimitadores \( y \) y el entorno math también pueden ser utilizados para escribir expresiones matemáticas en línea con el texto al igual que $ y $.

\(x\) y \(y\)

Podemos escribir expresiones más largas como $k_{n+1}=n^2 + k_n^2 - k_{n-1}$ ó $Y_{(ij)} = \mu + t_i + \epsilon  j (i) $
\end{document}