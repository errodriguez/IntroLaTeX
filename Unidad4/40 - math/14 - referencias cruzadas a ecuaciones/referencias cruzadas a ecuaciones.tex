\documentclass[10pt,letterpaper]{article}
\usepackage[utf8]{inputenc}
\usepackage{amsmath} %Para escribir ecuaciones no enumeradas y referencias a ecuaciones con paréntesis
\usepackage[hidelinks]{hyperref}
\title{Referencias cruzadas a ecuaciones}
\author{Curso de \LaTeX}
\begin{document}
\maketitle

Podemos hacer referencias cruzadas a las ecuaciones enumeradas, utilizando la instrucción \texttt{\textbackslash label} dentro del entorno \texttt{\textbackslash equation} como en el siguiente ejemplo:

Y después de experimentar mucho con diferentes técnicas resulta que la ecuación \ref{eq:sumatoria} es muy importante.
\begin{equation} \label{eq:sumatoria}
  w = \sum_{i=1}^{n} {(x_{i}+y_{i})^{2}}   
\end{equation}

... y como sabemos que
\begin{equation*} %Con el * indicamos que la ecuación no estará numerada (requiere el paquete amsmath)
  \lim_{x \to 0} {(x^{2} + 2x + 4)} = 4
\end{equation*}
se concluye que...

\vspace{0.5cm}

También podemos hacer referencia a varias ecuaciones contenidas dentro de un entorno \texttt{eqnarray} (funciona también con \texttt{align}):

\begin{eqnarray}
	c^2&=&a^2 + b^2\label{eq:teorema}\\
	c &=&\sqrt{a^2 + b^2}\label{eq:hipotenusa}
\end{eqnarray}

Como podemos observar, de la ecuación \ref{eq:teorema} (que representa el teorema de Pitágoras),  la hipotenusa puede despejarse como se muestra en la ecuación \ref{eq:hipotenusa}.

O dicho de otra forma, de \eqref{eq:teorema} se puede deducir \eqref{eq:hipotenusa}  para obtener el valor de la hipotenusa (lado mayor). %Nota: podemos ver que el comando \eqref hace la referencia incluyendo el número de la ecuación entre paréntesis, una forma pensada para hacer referencias específicas a ecuaciones (sin tener que mencionar la palabra ecuación)

Puedes revisar más comandos y ejemplos avanzados del entorno matemático y del paquete \textbackslash\texttt{amsmath} en la wiki: \url{http://en.wikibooks.org/wiki/LaTeX/Advanced_Mathematics}
\end{document}