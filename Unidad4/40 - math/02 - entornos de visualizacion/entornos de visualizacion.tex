\documentclass[12pt,letterpaper]{article}
\usepackage[utf8]{inputenc}
\usepackage[spanish]{babel}
\author{Curso de \LaTeX}
\title{Entornos de visualización para expresiones matemáticas}
\begin{document}
\maketitle

Hay tres entornos para escribir ecuaciones por fuera del texto.

\section{equation}

Se utiliza para escribir expresiones matemáticas (principalmente ecuaciones), que estarán enumeradas.

\begin{equation}
z = x + 2y
\end{equation}

\section{displaymath}

Similar al anterior, solo que la ecuación no estará enumerada. 

\begin{displaymath}
z = x^2 + 2y
\end{displaymath}

%Nota: los delimitadores \[ ... \] y $$ ... $$ pueden utilizarse como una abreviación de \begin{displaymath} ... \end{displaymath}

\section{eqnarray}

El resultado es similar a si escribiéramos la expresión matemática dentro de un entorno \texttt{tabular} de tres columnas. Es muy útil para escribir sistemas de ecuaciones lineales:

\begin{eqnarray}
a_1 & = & b_1 + c_1\nonumber\\
a_2 & = & b_2 - c_2 + 6
\end{eqnarray}

%Nota: el comando \nonumber se utiliza para prevenir que la primera línea del conjunto de ecuaciones aparezca enumerada, ya que ya que de otra forma \LaTeX enumeraría cada una de las ecuaciones.

\section{eqnarray*}

Si no queremos enumerar ninguna de las ecuaciones, también podemos utilizar el entorno alternativo \texttt{eqnarray*}:

\begin{eqnarray*}
	a_1 & = & b_1 + c_1\\
	a_2 & = & b_2 - c_2 + 6
\end{eqnarray*}

En matemáticas, a los sistemas de ecuaciones lineales se les asocian arreglos numéricos para buscar su solución, de ahí que el nombre de este entorno esté relacionado con los arreglos.
\end{document}