\documentclass[12pt,letterpaper]{article}
\usepackage[utf8]{inputenc}
\usepackage{graphicx} %Para importar gráficos
\usepackage[greek, spanish, mexico]{babel} %Para soportar texto en griego con el comando \greektext y que el contenido del documento esté en español de México
\usepackage[hidelinks]{hyperref} %Para el enlace al final del documento
\author{Tu nombre}
\title{}
\parskip=0.3cm
\begin{document}
\maketitle
El número Pi

Introducción

Pi (pi) es la relación entre la longitud de una circunferencia y su diámetro (ver figura 1), en geometría euclidiana. Es un número irracional y una de las constantes matemáticas más importantes. Se emplea frecuentemente en matemáticas, física e ingeniería. 

[Figura 1]
{Pi es obtenido por la relación Cd  = pi}

El valor numérico de Pi, truncado a sus primeras cifras, es el siguiente:

P ~~ 3.14159265358979323846

El valor de Pi se ha obtenido con diversas aproximaciones a lo largo de la historia, siendo una de las constantes matemáticas que más aparece en las ecuaciones de la física, junto con el número e. Cabe destacar que el cociente entre la longitud de cualquier circunferencia y la de su diámetro no es constante en geometrías no euclídeas.


El nombre $\Pi$

La notación con la letra griega $\Pi$ proviene de la inicial de las palabras de origen griego ``\textgreek{περιφέρεια}'' (periferia) y ``\textgreek{περίμετρον}'' (perímetro) de una círcunferencia, notación que fue utilizada primero por William Oughtred (1574-1660), y propuesto su uso por el matemático galés William Jones (1675-1749), aunque fue el matemático Leonhard Euler, con su obra ``Introducción al cálculo infinitesimal'' de 1748, quien la popularizó. Fue conocida anteriormente como constante de Ludolph (en honor al matemático Ludolph van Ceulen) o como constante de Arquímedes (que no se debe confundir con el número de Arquímedes).

[Figura 2]{Letra griega $\pi$. Símbolo adoptado en 1706 por William Jones y popularizado por Leonhard Euler.}


Historia del cálculo del valor Pi

El matemático griego Arquímedes (siglo III a. C.) fue capaz de determinar el valor de Pi, entre el intervalo comprendido por 3 10/71, como valor mínimo, y 3 1/7, como valor máximo. Con esta aproximación de Arquímedes se obtiene un valor con un error que oscila entre 0.024\% y 0.040\% sobre el valor real. El método usado por Arquímedes era muy simple y consistía en circunscribir e inscribir polígonos regulares de n-lados en circunferencias y calcular el perímetro de dichos polígonos. Arquímedes empezó con hexágonos circunscritos e inscritos, y fue doblando el número de lados hasta llegar a polígonos de 96 lados.

[figura 3]
{Método de Arquímedes para encontrar dos valores que se aproximen al número Pi, por exceso y defecto.}



Alrededor del año 20 d. C., el arquitecto e ingeniero romano Vitruvio calcula Pi como el valor fraccionario 25/8 midiendo la distancia recorrida en una revolución por una rueda de diámetro conocido.

En el siglo II, Claudio Ptolomeo proporciona un valor fraccionario por aproximaciones:

pi ~- 377 120 = 3.1416

Mas información en: \url{http://es.wikipedia.org/wiki/N\%C3\%BAmero\_\%CF\%80}
\end{document}