\documentclass[12pt,letterpaper]{article}
\usepackage[utf8]{inputenc}
\usepackage{chemfig}
\author{Curso de \LaTeX}
\title{Anillos}
\begin{document}
\maketitle

Los anillos siguen la sintaxis $ < $átomo$ > $*$ < $n$ > $(código), donde ``n'' indica el número de caras del anillo y ``código'' representa el contenido específico de cada anillo (enlaces y átomos).

\begin{center}
\chemfig{A*6(-B-C-D-E-F-)}

\chemfig{A*5(-B-C-D-E-)}

\chemfig{*6(=-=-=-)}

\chemfig{**5(-----)}
\end{center}

Cuando una molécula no es iniciada por un anillo y uno o más enlaces ya han sido dibujados, la posición por defecto del ángulo cambia: el anillo es dibujado de tal manera que el enlace que termina en el átomo de unión biseca el ángulo formado por los lados primero y último del anillo.

\begin{center}
\chemfig{A-B*5(-C-D-E-F-)}
\end{center}

Esta regla se mantiene, sin importar el ángulo del enlace precedente:

\begin{center}
\chemfig{A-[:25]B*4(----)}\vskip5pt
\chemfig{A=[:-30]*6(=-=-=-)}
\end{center}

Para colocar enlaces alrededor del anillo colocamos un paréntesis después de cada uno de los enlaces internos del anillo:

\begin{center}
\chemfig{*6(-(-R^2)=-(-)=(-OH)-(-R^1)=)}
\end{center}

De esta forma, podemos unir dos anillos por medio de enlaces, colocando el nuevo anillo al interior de uno de los enlaces del primer anillo:

\chemfig{
  *6(-(-R^2)=-
    (-=N-*6(=(-R^3)-=(-R^4)-=(-R^3)-))
  =(-OH)-(-R^1)=)
}
\end{document}