\documentclass[10pt,letterpaper]{article}
\usepackage[utf8]{inputenc}
\title{Introducing to \LaTeX}
\author{Curso de introducción a LaTeX}
\begin{document}
\maketitle  
\begin{abstract}
\LaTeX\ is a document preparation system for the \TeX
typesetting program. It offers programmable desktop publishing features and extensive facilities for automating most aspects of
typesetting and desktop publishing, including numbering and
cross-referencing, tables and figures, page layout, bibliographies,
and much more. \LaTeX was originally written in 1984 by Leslie
Lamport and has become the dominant method for using \TeX; few
people write in plain \TeX anymore. The current version is
\LaTeXe.
\end{abstract}
\renewcommand{\abstractname}{Resumen} %Cambiamos el texto del encabezado del abstract
\begin{abstract}
\LaTeX\ es un sistema de elaboración de documentos para el lenguaje de composición tipográfica \TeX{}. Éste ofrece opciones de autoedición programable y gran comodidad para la automatización de la mayoría de los aspectos de la composición tipográfica y autoedición, incluyendo numeración y referencias cruzadas, tablas y figuras, diseño de página, bibliografías, y mucho más. \LaTeX{} fue escrito originalmente en 1984 por Leslie
Lamport y se ha convertido en el método dominante para utilizar \TeX{}; ya poca gente escribe en \TeX{} plano. La versión actual es \LaTeXe{}.
\end{abstract}

What is TeX?

TeX is a low-level markup and programming language created by Donald Knuth to typeset documents attractively and consistently. Knuth started writing the TeX typesetting engine in 1977 to explore the potential of the digital printing equipment that was beginning to infiltrate the publishing industry at that time, especially in the hope that he could reverse the trend of deteriorating typographical quality that he saw affecting his own books and articles. With the release of 8-bit character support in 1989, TeX development has been essentially frozen with only bug fixes released periodically. TeX is a programming language in the sense that it supports the if-else construct: you can make calculations with it (that are performed while compiling the document), etc., but you would find it very hard to do anything else but typesetting with it. The fine control TeX offers over document structure and formatting makes it a powerful and formidable tool. TeX is renowned for being extremely stable, for running on many different kinds of computers, and for being virtually bug free. The version numbers of TeX are converging toward $\pi$, with a current version number of 3.1415926.
\end{document}