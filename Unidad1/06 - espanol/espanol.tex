\documentclass[10pt,letterpaper]{article}
\usepackage[utf8]{inputenc}
\usepackage[spanish, mexico]{babel}
\title{Háblame en español}
\author{Curso de introducción a LaTeX}
\begin{document}
\maketitle
\begin{abstract}
La palabra ``abstract'' la pone \LaTeX automáticamente. Si queremos poner otra cosa, por ejemplo ``resumen'', incluimos en el preámbulo la instrucción \textbackslash renewcommand\{\textbackslash abstractname\}\{resumen\}. Otra alternativa, es cambiar el idioma completo del documento, para que \LaTeX traduzca todas los encabezados que genera automáticamente a ese idioma en concreto. Para lograrlo, utilizamos el paquete babel con la opción \texttt{spanish}.
\end{abstract}
\end{document}